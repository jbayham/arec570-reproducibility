\documentclass[]{article}
\usepackage{lmodern}
\usepackage{amssymb,amsmath}
\usepackage{ifxetex,ifluatex}
\usepackage{fixltx2e} % provides \textsubscript
\ifnum 0\ifxetex 1\fi\ifluatex 1\fi=0 % if pdftex
  \usepackage[T1]{fontenc}
  \usepackage[utf8]{inputenc}
\else % if luatex or xelatex
  \ifxetex
    \usepackage{mathspec}
  \else
    \usepackage{fontspec}
  \fi
  \defaultfontfeatures{Ligatures=TeX,Scale=MatchLowercase}
\fi
% use upquote if available, for straight quotes in verbatim environments
\IfFileExists{upquote.sty}{\usepackage{upquote}}{}
% use microtype if available
\IfFileExists{microtype.sty}{%
\usepackage{microtype}
\UseMicrotypeSet[protrusion]{basicmath} % disable protrusion for tt fonts
}{}
\usepackage[margin=1in]{geometry}
\usepackage{hyperref}
\hypersetup{unicode=true,
            pdftitle={GDP and Life Expectancy},
            pdfauthor={Jude Bayham},
            pdfborder={0 0 0},
            breaklinks=true}
\urlstyle{same}  % don't use monospace font for urls
\usepackage{graphicx,grffile}
\makeatletter
\def\maxwidth{\ifdim\Gin@nat@width>\linewidth\linewidth\else\Gin@nat@width\fi}
\def\maxheight{\ifdim\Gin@nat@height>\textheight\textheight\else\Gin@nat@height\fi}
\makeatother
% Scale images if necessary, so that they will not overflow the page
% margins by default, and it is still possible to overwrite the defaults
% using explicit options in \includegraphics[width, height, ...]{}
\setkeys{Gin}{width=\maxwidth,height=\maxheight,keepaspectratio}
\IfFileExists{parskip.sty}{%
\usepackage{parskip}
}{% else
\setlength{\parindent}{0pt}
\setlength{\parskip}{6pt plus 2pt minus 1pt}
}
\setlength{\emergencystretch}{3em}  % prevent overfull lines
\providecommand{\tightlist}{%
  \setlength{\itemsep}{0pt}\setlength{\parskip}{0pt}}
\setcounter{secnumdepth}{0}
% Redefines (sub)paragraphs to behave more like sections
\ifx\paragraph\undefined\else
\let\oldparagraph\paragraph
\renewcommand{\paragraph}[1]{\oldparagraph{#1}\mbox{}}
\fi
\ifx\subparagraph\undefined\else
\let\oldsubparagraph\subparagraph
\renewcommand{\subparagraph}[1]{\oldsubparagraph{#1}\mbox{}}
\fi

%%% Use protect on footnotes to avoid problems with footnotes in titles
\let\rmarkdownfootnote\footnote%
\def\footnote{\protect\rmarkdownfootnote}

%%% Change title format to be more compact
\usepackage{titling}

% Create subtitle command for use in maketitle
\providecommand{\subtitle}[1]{
  \posttitle{
    \begin{center}\large#1\end{center}
    }
}

\setlength{\droptitle}{-2em}

  \title{GDP and Life Expectancy}
    \pretitle{\vspace{\droptitle}\centering\huge}
  \posttitle{\par}
    \author{Jude Bayham}
    \preauthor{\centering\large\emph}
  \postauthor{\par}
      \predate{\centering\large\emph}
  \postdate{\par}
    \date{October 15, 2019}


\begin{document}
\maketitle

\hypertarget{canada}{%
\subsection{Canada}\label{canada}}

Figure 1 shows that life expectancy has been increasing over time in
Canada. That is good.

Figure 2 shows that Canad's GDP per capita is also increasing over time.
Go Canada!

I wonder if these two data are correlated?

\includegraphics{manuscript_files/figure-latex/canada_le-1.pdf}

\includegraphics{manuscript_files/figure-latex/canada_gdp-1.pdf}

\hypertarget{life-expectancy-and-gdp}{%
\subsection{Life Expectancy and GDP}\label{life-expectancy-and-gdp}}

The objective of this analysis is to test whether GDP per capita leads
to higher life expectancy. We hypothesize that the relationship between
GDP per capita and life expectancy is positive. We regress life
expectancy from 140 countries on GDP per capita to test our hypothesis.
The results are displayed in Table 1.

\begin{table}[!htbp] \centering 
  \caption{GDP on Life Expectancy} 
  \label{} 
\begin{tabular}{@{\extracolsep{5pt}}lc} 
\\[-1.8ex]\hline 
\hline \\[-1.8ex] 
 & \multicolumn{1}{c}{\textit{Dependent variable:}} \\ 
\cline{2-2} 
\\[-1.8ex] & lifeExp \\ 
\hline \\[-1.8ex] 
 gdpPercap & 0.0005$^{***}$ \\ 
  & (0.0001) \\ 
  & \\ 
 Constant & 50.225$^{***}$ \\ 
  & (0.540) \\ 
  & \\ 
\hline \\[-1.8ex] 
Observations & 568 \\ 
R$^{2}$ & 0.130 \\ 
Adjusted R$^{2}$ & 0.128 \\ 
Residual Std. Error & 11.472 (df = 566) \\ 
F Statistic & 84.367$^{***}$ (df = 1; 566) \\ 
\hline 
\hline \\[-1.8ex] 
\textit{Note:}  & \multicolumn{1}{r}{$^{*}$p$<$0.1; $^{**}$p$<$0.05; $^{***}$p$<$0.01} \\ 
\end{tabular} 
\end{table}

The coefficient on gdpPercap is positive and statistically significant
at \(\alpha=0.05\), which indicates that for every \$10,000 increase in
GDP per capita, life expectancy increases by five years.

We can conclude that by continually increasing GDP we can increase life
expectancy.

Further research should focus on the nonlinear relationship between life
expectancy and GDP per capita.


\end{document}
